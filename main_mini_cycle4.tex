%% LaTeX template for the science justification & technical
%% feasibility to be submitted as part of a TESS Guest Investigator
%% Program proposal. This template is based on the proposal template
%% used by the NuSTAR mission.
%%
%% TESS Guest Investigator Proposal Cycle 4 template
%% V1.0
%% 2017-08-04
%% V1.1
%% 2019-02-07
%% V1.2
%% 2019-10-27
%% V1.3
%% 2020-11-02

%%%%%%%%%%%%%%%%%%%%%%%%%%%
%%%%% DOCUMENT FORMAT %%%%%
%%%%%%%%%%%%%%%%%%%%%%%%%%%

%% The default font was chosen to be easily readable while allowing
%% sufficient material to be included.

%% Please note that the proposal will be printed on US Letter size paper,
%% 8.5 in x 11 in, and that formatting the text for other sizes will
%% generally cause layout problems and may result in text being cut
%% off near the edges. PLEASE DO NOT CHANGE THE 'LETTERPAPER' OPTION
%% IN THE DOCUMENTCLASS COMMAND.

%%%%%%%%%%%%%%%%%%%%%%%%%%%%%%%%%%%%%%%%%%%%%%
%%%%% Default format: 12pt single column %%%%%
%%%%%%%%%%%%%%%%%%%%%%%%%%%%%%%%%%%%%%%%%%%%%%


%% Minimum margin size is 1 inch from top, bottom, and sides.
%% Font size: see NASA Guidebook for Proposers 
%%(https://www.hq.nasa.gov/office/procurement/nraguidebook/proposer2018.pdf).

\documentclass[letterpaper,12pt]{article}

%%%%%%%%%%%%%%%%%%%%%%%%%%%%%%%%%%
%%%%% HOW TO INCLUDE FIGURES %%%%%
%%%%%%%%%%%%%%%%%%%%%%%%%%%%%%%%%%

%% Please see the ``Included packages'' section below.

%%%%%%%%%%%%%%%%%%%%%%%%%%%%%
%%%%% Included packages %%%%%
%%%%%%%%%%%%%%%%%%%%%%%%%%%%%

\usepackage{graphics,graphicx}
\usepackage[colorlinks]{hyperref}
\hypersetup{urlcolor=blue}
\usepackage{cleveref}
\usepackage{natbib}

%% Feel free to modify the included packages list to use your
%% favorite packages. 

%% In the graphics and graphicx packages, Postscript and eps figures
%% can be included using the \includegraphics command. The graphics
%% package is part of standard LaTeX2e and provides a basic way of including a
%% figure. The graphicx package is not standard, but extends the
%% \includegraphics command to make it more user-friendly. If graphicx
%% is not available on your system please remove it from the list of
%% included packages above.  

%% Syntax:
%% In the graphics package:
%%
%% \begin{figure}
%% \includegraphics[llx,lly][urx,ury]{file}
%% \end{figure}
%%
%% where ll denotes 'lower left' and ur 'upper right' and the x and y
%% values are the coordinates of the PostScript bounding box in
%% points. There are 72 points in an inch.
%%
%% In the graphicx package:
%% 
%% \begin{figure}
%% \includegraphics[key=val,key=val,...]{file}
%% \end{figure}
%%
%% where some of the useful keys are: angle, width, height,
%% keepaspectratio (='true' or 'false') and scale. Bounding box values
%% can be given as [bb=llx lly urx ury].
%%
%% In either case you have to use LaTeX figure placement commands to
%% position the figure on the page; \includegraphics will not do
%% that. Both these commands also have other options that are listed
%% in the LaTeX manual (for the graphics package) and in 'The LaTeX
%% Graphics Companion' (for the graphicx package).



%%%%%%%%%%%%%%%%%%%%%%%%%%%
%%%%% Page dimensions %%%%%
%%%%%%%%%%%%%%%%%%%%%%%%%%%

\setlength{\textwidth}{6.5in} 
\setlength{\textheight}{9in}
\setlength{\topmargin}{-0.0625in} 
\setlength{\oddsidemargin}{0in}
\setlength{\evensidemargin}{0in} 
\setlength{\headheight}{0in}
\setlength{\headsep}{0in} 
\setlength{\hoffset}{0in}
\setlength{\voffset}{0in}



%%%%%%%%%%%%%%%%%%%%%%%%%%%%%%%%%%
%%%%% Section heading format %%%%%
%%%%%%%%%%%%%%%%%%%%%%%%%%%%%%%%%%

\makeatletter
\renewcommand{\section}{\@startsection%
{section}{1}{0mm}{-\baselineskip}%
{0.5\baselineskip}{\normalfont\Large\bfseries}}%
\makeatother

%%%%%%%%%%%%%%%%%%%%%%%%%%%%%%%%%%%%%
%%%%% Some Useful Abbreviations %%%%% 
%%%%%%%%%%%%%%%%%%%%%%%%%%%%%%%%%%%%%
\newcommand{\tess}{{\it TESS}}
\newcommand{\jwst}{{\it JWST}}
\newcommand{\kepler}{{\it Kepler}}
\newcommand{\ktwo}{{K2}}
\newcommand{\hst}{{\it HST}}
\newcommand{\swift}{{\it Swift}}
\newcommand{\integral}{{\it INTEGRAL}}
\newcommand{\nustar}{{\it NuSTAR}}
\newcommand{\fermi}{{\it Fermi}}
\newcommand{\ms}{$M_{\odot}$}
\newcommand{\rs}{$R_{\odot}$}
\newcommand{\ls}{$L_{\odot}$}
\newcommand{\re}{$R_{\oplus}$}
\newcommand{\me}{$M_{\oplus}$}
\newcommand{\kms}{km~s$^{-1}$}
\newcommand{\fluxcgs}{ergs~s$^{-1}$~cm$^{-2}$}
\newcommand{\lumcgs}{ergs~s$^{-1}$}
\newcommand{\rj}{$R_{\textrm{\scriptsize Jup}}$}
\newcommand{\mj}{$M_{\textrm{\scriptsize Jup}}$}
\newcommand{\ms}{m~s$^{-1}$}


%%%%%%%%%%%%%%%%%%%%%%%%%%%%%
%%%%% Start of document %%%%% 
%%%%%%%%%%%%%%%%%%%%%%%%%%%%%

\begin{document}
\pagestyle{plain}
\pagenumbering{arabic}


 
%%%%%%%%%%%%%%%%%%%%%%%%%%%%%
%%%%% Title of proposal %%%%% 
%%%%%%%%%%%%%%%%%%%%%%%%%%%%%

\begin{center} 
\bfseries\uppercase{%
%%
%% ENTER TITLE OF PROPOSAL BELOW THIS LINE
MONITORING THE JWST SPECTRO-PHOTOMETRIC STANDARDS
%%
%%
}
\end{center}



%%%%%%%%%%%%%%%%%%%%%%%%%%%%%%%%%%%%%%%%%
%%%%% Body of science justification %%%%%
%%%%% and technical feasibility     %%%%%
%%%%%%%%%%%%%%%%%%%%%%%%%%%%%%%%%%%%%%%%%


%%%%%%%%%%%%%%%%%%%%%%%%%%%%%%%%%%%%%%%%%
%%%%%%%%%%%%%%%%%%%%%%%%%%%%%%%%%%%%%%%%%
%%%%% The text below should be commented out before submitting your proposal %%%%% 
% \noindent{The recommended sections for a \tess\ GI proposal are shown below. Feel free to change section 
% headings as necessary, but this is the suggested minimal information that should be included in the proposal. 
% This Science/Technical section of the proposal is limited to 2 pages. Figures are included in these page limits, but not references or a (sample) target table. \\

% \noindent Note that the Phase-1 proposal review will be done in a dual-anonymous fashion and follow the guidelines listed below:
% \begin{itemize}
%     \item Proposals should eliminate language that identifies the proposers or institution, as discussed in the \href{https://science.nasa.gov/researchers/dual-anonymous-peer-review}{Guidelines for Anonymous Proposals}.
%     \item PIs are required to upload a one-page Team Expertise (insert link) PDF through a separate upload when submitting the science justification into ARK/RPS.
%     \item Proposals that do not follow these dual-anonymous guidelines may be returned without review.
% \end{itemize}

% \noindent Mini proposals are intended for requests for a small number of target slots and require minimal resources, up to 50 20-second cadence targets and 1,000 2-minute cadence targets. Proposals in this category are not eligible for funding.\\ 

% \noindent Mini proposals cannot have Targets of Opportunity, a joint component with \hst, \swift, or \fermi, or have a ground-based component. 
%  } 
%%%%% The text above should be commented out before submitting your proposal %%%%% 
%%%%%%%%%%%%%%%%%%%%%%%%%%%%%%%%%%%%%%%%%
%%%%%%%%%%%%%%%%%%%%%%%%%%%%%%%%%%%%%%%%%



\section{Abstract}
We propose to use the TESS 2-minute cadence to monitor the James Webb Space Telescope (\jwst) spectro-photometric standard stars for unexpected variability. These stars are necessary to accurately calibrate the absolutely and relative flux \jwst\ receives across its infrared wavebands. The standards used by \jwst\ include white dwarf stars, A stars and G giant stars. By monitoring these stars at a 2-minute cadence we will be able to determine if the star has recently shown evidence of brightness variations due to flares, pulsations or stellar spots. Additionally, we will also be able to check that it has not recently undergone occultations from planets or dust \citep[][]{Boyajian2016}, long term variations like recently seen in Betelgeuse \citep{Guinan2019}, or shows evidence of stellar activity \citep{Kohler2017nova}. By monitoring these stars with \tess\ at a short cadence we add to a wealth of knowledge about these stars and enable the \jwst\ team to check for certain astrophysical scenarios as they attempt to achieve the missions required 2\% photometry \citep{jdox}.


 %REMINDER:  THIS NEEDS to be written so that no one knows who we ARE
\section{Scientific Justification and Perceived Impact}

JWST is planning to launching in October 2021 and will likely revolutionize our view of all areas of astrophysics from Galaxy formation to exoplanets.As it begins to collect science data, the mission will also observe a few dozen spectro-photometric standard stars in order to calibrate all science data collected by this mission. The nominal requirement is 2\% absolute flux prediction accuracy of standard stars, with the goal of improving it as much as possible.  To achieve this, a sample of stars will well known absolute flux and spectral shape across the 0.6 to 28.5 microns range will be used. The success of this unified program will not only enable the cross-instrument calibration of the \jwst\ science instruments, but also with \hst, \spitzer, and other ground-based telescopes.


Many of the scientific programs that have been proposed for JWST require this accurate calibration across instruments and telescopes.  For example, those looking for evidence of a dust disk from infrared excesses will want to compare the photometry across all of the JWST bands.  Something with Galaxies....something with exoplanets....\textit{Short paragraph on science programs that require accurate photometry to emphasize the impact}

The JWST calibration program is performed by observing NN stars with various instruments ... \textit{Describe how photometric standards are used emphasizing the our assumptions about photometric stability}

TESS and Kepler have taught us that stars can unexpectedly change in brightness due to a variety of reasons including flares, stellar pulsations, spots and occultations. Recent examples of stars changing flux in unexpected ways include:
\begin{itemize}
    \item  The star KIC~8462852 \citep{Boyajian2016} was discovered by Kepler to show still unexplained drops in flux as large as 20\% at a periodic times.
    \item The nearby, giant star Betelgeuse \citep{Guinan2019} dimmed by more than 1 visual magnitude in 2019, a dimming larger than had ever been reported in 50 plus years of observations \citep{Levesque2020ApJ}.
    \item The pulsating white dwarf star, GD~358, suddenly changed the period and amplitude of its pulsations \citep{Montgomery2010}. In 1996, GD 358 was observed to move all of its pulsation energy into a single pulsation mode with amplitudes as large as 2\%. White dwarfs have been discovered to in a variety of ways \citep{Hermes2017} due to magnetic fields, pulsations and binarity.
    \item Some white dwarf stars have been discovered to show sudo-periodic dips in brightness as large as 40\% that last approximately 8 minutes. These are believed to be caused by a set of disintegrating minor planets \citep{Vanderburg2015}.
\end{itemize}

These stories are becoming more common as \tess\ has been able to monitor the variability of the night sky and teach us that there are a variety of reasons, both known and unknown that cause stars to vary in brightness. By monitoring these stars in TESS Cycle 4, we will have more information in case unexpected flux changes occur on these standard stars. If unusual variations are detected, it will enable the \jwst\ calibration team to determine if it will impact their ability to use the stars as a photometric standard. 

We request two-minute cadence monitoring for these stars.  The exposure times used in the JWST calibration plan are on the order of ??several minutes??? and so we are most interested in variability that is longer than this time scale.  Also, most known, large variability of stars can be detected at the 2 minute cadence, e.g. flares, pulsations, occultation and stellar spots. 

To further emphasize the importance of this sort of monitoring, an initial review of existing TESS data for these stars has already revealed pulsations with periods of several minutes on two of the spectro-photometric standards. While the amplitudes of the pulsations are likely not enough to impact the JWST calibrations, these TESS observations will help us better understand possible astrophysics that can impact its ability to act as a spectro-photometric standard.
Should we ADD A FIGURE?? I doubt we will find room.


%Provide text and figures that justify the scientific need for new \tess\ observations and analyses here. In particular, justify your choice of new 2~min or 20~s cadence observations. If you will also be making use of the 10~min FFIs for your research, make it clear why the \tess\ FFI data are suitable for your science.

%Summarize the expected science return of the proposed investigations and the expected benefit to the community.


\section{Analysis Plan and Technical Feasibility}
Discuss how many targets we can observe in Cycle 4. How bright they are? SNR expected. Number of sectors they are observable in.
Mention that the observations will not necessarily be concomitant with the JWST observations, but they will still be useful. 

We plan to analyze the variability of these targets using the lightkurve software package. We will look for both periodic variability and sudden changes in brightness. Periodic detections will be found by taking a periodogram and looking for significant peaks. Single events will be discovered by visual inspection of the light curve. We will report the limits of any detected variability, or indeed if any variability is detected, in a refereed paper and \jwst\ technical note.


%Discuss how you plan to analyze the \tess\ data. Provide text and figures showing that the proposed \tess\ investigations are feasible; consider the 
%\tess\ survey strategy, target observability, and required signal-to-noise, etc. The \tess\ Science 
%Support Center (\href{https://heasarc.gsfc.nasa.gov/docs/tess/}{TSSC}) makes several tools available to help estimate these quantities. 

\clearpage

\bibliography{references}{}
\bibliographystyle{abbrv}
%List of references. References {\it are {\bf not} included} when considering the proposal page limit. References in the text should be in the number format, and in the references list as:

%[1] Person A, Person B, Person C, et al., 2016, ApJ 200, 231, 2\\
%[2] Person D \& Person E, 1912, Nature 495, 452


\section{Target Table}

When necessary to justify your proposal, provide a list of targets using the below example as a template for format. This target table is designed to aid reviewers and need only provide a representative sample of the complete target list uploaded to RPS. Full target tables should be submitted electronically with the Phase-1 proposal. Please limit any target table included here to only 1 page. The table is not included in the page limit of the Science/Technical section. 


\begin{center}
\begin{tabular}{ | c | c | c | c | c | c | c | }
\hline
TIC ID          &      Common      &     RA             &      Dec          &      TESS       &       Obj.        &      Comments \\       
                    &      Name           &     (deg)          &      (deg)        &      mag         &       Type       &                         \\     
\hline
\hline
388857263  &  Prox Cen           &  217.428793  &  -62.679592  &  7.36             &    M Dwarf    & 2 min cad., RV planet \\ \hline
353622691  &  BL Lac               &   330.6803807    &   42.2777717    &   13.1  &   AGN            &    20 s cad.                                 \\ \hline
                    &                           &                       &                      &                      &                     &                                     \\ \hline
                    &                           &                       &                      &                      &                     &                                     \\ \hline
                    &                           &                       &                      &                      &                     &                                     \\ \hline
\end{tabular}
\end{center}   

%%%%%%%%%%%%%%%%%%%%%%%%%%%
%%%%% End of document %%%%%
%%%%%%%%%%%%%%%%%%%%%%%%%%%

\end{document}

