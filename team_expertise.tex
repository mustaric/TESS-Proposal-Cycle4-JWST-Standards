Expertise and contribution of team members:

\textbf{Susan Mullally, Deputy Project Scientist for JWST at STScI.} Dr. Mullally has experience working with time series data. She was previously a scientist for the Kepler mission and archive scientists for the TESS mission at MAST.  She has studied time variability using the Whole Earth Telescope, Kepler and TESS to find and study objects like pulsating white dwarf stars, eccentric binary stars, and transiting exoplanets.  Dr. Mullally will lead the effort to analyze the data that results from this program.


Karl Gordon


Greg Sloan


\textbf{Catherine Kaleida, Senior Staff Astronomical Data Scientist and Deputy Lead for the JWST Data Management System Development at  STScI.}  Dr. Kaleida previously led a project to train students to take observations of Targets of Opportunity, including novae, supernovae, and Near Earth Objects such as Potentially Hazardous Asteroids, using the SMARTS 0.9-m Telescopes at the Cerro Tololo Inter-American Observatory.  She also has expertise in the Data Management System for JWST, and in star clusters and stellar associations in nearby galaxies.  Dr. Kaleida will assist in the effort to analyze the data that results from this program, and identify what additional observations would be needed from ground-based observatories to cover variability ranges not provided by the TESS data from this program.

J. J. Hermes

Kelly Hambleton

