Expertise and contribution of team members -- Used to create attached PDF.

\textbf{Susan Mullally, Deputy Project Scientist for JWST at STScI.} Dr. Mullally has experience working with time series data. She was previously a scientist for the Kepler mission and archive scientists for the TESS mission at MAST.  She has studied time variability using the Whole Earth Telescope, Kepler and TESS to find and study objects like pulsating white dwarf stars, eccentric binary stars, and transiting exoplanets.  Dr. Mullally will lead the effort to analyze the data that results from this program.


\textbf{Karl Gordon} is the lead of the JWST absolute flux calibration program for all five instruments (NIRCam, NIRSpec, NIRISS, FGS, and MIRI).  Dr. Gordon has worked on data reduction and absolute calibration for a number of years including (in addition to JWST) the mid- and far-infrared instrument MIPS on the Spitzer Space Telescope.  He will coordinate the impact of the results on the overall JWST absolute flux calibration program.


\textbf{Greg Sloan} is the calibration lead for MIRI on JWST and lead for coordinating the Cycle 1 Calibration for all JWST instruments.  Dr. Sloan has decades of experience with infrared spectroscopy and led the spectrophotometric calibration of the Infrared Spectrograph on the Spitzer Space Telescope at Cornell.  He previously worked with infrared spectra from the Infrared Space Observatory and the Infrared Astronomical Satellite.  He will help determine how the JWST calibration will respond to any reported variability in it spectrophotometric standards.


\textbf{Catherine Kaleida, Senior Staff Astronomical Data Scientist and Deputy Lead for the JWST Data Management System Development at  STScI.}  Dr. Kaleida is familiar with ground-based follow-up and has lead students in projects to observe Targets of Opportunity, including novae, supernovae, and Near Earth Objects, using the SMARTS 0.9-m Telescopes at the Cerro Tololo Inter-American Observatory.  She has expertise in the Data Management System for JWST including the JWST pipeline. Dr. Kaleida will assist in the effort to analyze the data that results from this program, and will lead efforts to obtain additional observations from ground-based observatories to cover variability ranges not provided by the TESS data from this program.

\textbf{J. J. Hermes, Assistant Professor at Boston University.} Dr. Hermes is an expert in white dwarf stars and their use as photometric standards.  

\textbf{Kelly Hambleton, Assistant Professor at Villanova University.} Dr. Hambleton is an expert is A star variability and in time series data reduction. She is also the primary contact for pulsating stars with LSST.

